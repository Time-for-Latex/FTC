\documentclass[10pt,a4paper,landscape]{article}
\usepackage[utf8]{inputenc}
\usepackage{amsmath}
\usepackage{amsfonts}
\usepackage{amssymb}
\usepackage[shortlabels]{enumitem}
\usepackage[left=2cm,right=3.5cm,top=3cm,bottom=2.5cm]{geometry}
\author{Homenique Viera Martins}
\title{Lista 02 - FTC}

\begin{document}
\maketitle{}


\section{Dê a definição recursiva do conjunto de strings sobre o alfabeto \(\left\{  a, b  \right\}\) que contenha um número par de b´s.}


\section{Mostre que: }
    \begin{enumerate}[label=\Alph*]
        \item \((ba) + (a*b* U a*) = (ba)* ba+ (b* U \lambda)\) :
        \item \(b^+(a*b U \lambda )b = b(b*a* U \lambda) b^+ \) :
    \end{enumerate} 

\section{Forneça as expressões regulares para o conjunto de strings sobre:}
\begin{enumerate}[label=\Alph*]
    %uncomment A
    \item \(\sum = \left\{  a, b  \right\}\) de tamanho \(\ge 2\), no qual todos os a’s precedem todos os b’s.
    %uncomment B 
    \item \(\sum = \left\{  a, b  \right\}\) que contém o substring aa.
    %uncomment C
    \item \(\sum = \left\{  a, b  \right\}\) que possui exatamente um par  aa.
    %uncomment D
    \item \(\sum = \left\{  a, b  \right\}\) que começa com a, contém exatamente dois b’s e termina com cc.
    %uncomment E
    \item \(\sum = \left\{  a, b  \right\}\) que contém o substring  ab  e  o substring ba.
    %uncomment F
    \item \(\sum = \left\{  a, b, c  \right\}\) que contém o substring aa, bb e cc.
    %uncomment G
    \item \(\sum = \left\{  a, b, c  \right\}\) no qual cada b é imediatamente seguido por pelo menos um c.
    %uncomment H
    \item \(\sum = \left\{  a, b, c  \right\}\) de tamanho 3.
    %uncomment I
    \item \(\sum = \left\{  a, b, c  \right\}\) com tamanho menor que 3.
    %uncomment J
    \item \(\sum = \left\{  a, b, c  \right\}\) com tamanho maior que 3.
    %uncomment K
    \item \(\sum = \left\{  a, b  \right\}\) com um número par de a`s e impar de b´s.
\end{enumerate} 

\end{document}